\documentclass[11pt,b5paper,oneside,lualatex]{ltjsarticle} % LuaLaTeXの場合
%\documentclass[11pt,a4paper,oneside,titlepage,lualatex]{ltjsreport} % 表紙付き, 章から始まる形式

%SumatraPDFの逆順検索でエラーが出た時は以下のコマンドラインをSumatraPDFの設定→オプションで入力する
%"C:\Program Files (x86)\TeXstudio\texstudio.exe" "%f" -line %l

\usepackage{luatexja} % ltjclasses, ltjsclasses を使うときはこの行不要
\usepackage[marginparwidth=0pt,margin=10truemm]{geometry} % 余白の設定
% スマホやタブレットでも読みやすいB5サイズの文書を書くときは, 1行目の\documentclassのオプションで「a4paper」を「b5paper」にして, 余白設定はmargin=10truemmにすると自分好み

% --------------------------------------------------------------------------
%		パッケージとコマンド
% --------------------------------------------------------------------------

\usepackage{mypackage} % よく使うパッケージ. 「C:\w32tex\share\texmf-local\tex\(好きなファイル名)」に置いたmypackage.styを読み込む
\usepackage{mycommand} % 自分で定義したコマンド. 「C:\w32tex\share\texmf-local\tex\(好きなファイル名)」に置いたmycommand.styを読み込む

\usetikzlibrary{knots}

%\usetikzlibrary{graphs,graphs.standard,graphdrawing} % TikZでグラフを描く
%\usegdlibrary{trees,force,layered} %graphdrawingの子ライブラリ

\DeclareMathOperator{\WRT}{WRT}
\DeclareMathOperator{\perm}{perm}

% --------------------------------------------------------------------------
%		ハイパーリンク
% --------------------------------------------------------------------------

%目次にもハイパーリンクが付く. プリアンブルのできるだけ後ろに書く. 
\usepackage[luatex, pdfencoding=auto,hypertexnames=false]{hyperref}
\hypersetup{% hyperrefオプションリスト
	colorlinks=true,
	linkcolor=DarkGoldenrod, % リンクの色
	citecolor=SlateBlue, % 引用文献の色
	urlcolor=violet, % URLの色
}

% --------------------------------------------------------------------------
%		定理環境と相互参照
% --------------------------------------------------------------------------

% 参照番号の設定
\numberwithin{equation}{section} % 式番号
\newtheorem{theoremcounter}{}[section] % 定理番号のオプションを選択. [section]を[chapter]にすれば章番号から始まり「定理 1.1.1」のようになる.
\newtheorem{exercisecounter}{}[] % 演習問題番号のオプションを選択.

% 自作スタイルファイル読み込み
\usepackage{mytheorem} % cleverefパッケージによる定理環境と相互参照. 「C:\w32tex\share\texmf-local\tex\(好きなファイル名)」に置いたmytheorem.styを読み込む. \usepackage{hyperref}の後に書く. 

%\usepackage{myprogram} % ハイライト付きソースコード. 「C:\w32tex\share\texmf-local\tex\(好きなファイル名)」に置いたmyprogram.styを読み込む

% --------------------------------------------------------------------------
\begin{document}
% --------------------------------------------------------------------------

\title{Zwegersの不定値テータ関数}
\author{村上友哉}
\date{\today}

\maketitle

%シンプルな目次
\tableofcontents

% --------------------------------------------------------------------------

\section{導入と主結果} \label{sec:intro}

% --------------------------------------------------------------------------



本稿の構成を述べる. 
\cref{sec:fund_data}では本稿を通して用いる記号やHグラフに関する基本的な性質を述べる. 
\cref{sec:Gauss_sum}では本稿で必要なGauss和に関する性質をまとめる. 
\cref{sec:asymptotic_formula}では漸近展開の公式を導出する. 
\cref{sec:Zwegers_theta}ではZwegersの不定値テータ関数について考察する. 
\cref{sec:infinite_series}ではWRT不変量を極限値に持つような無限級数を新しく定義する. 
\cref{sec:examples}では本稿で考察するHグラフの具体例と, 本稿の議論から得られるPoincar\'{e}ホモロジー球面の性質を述べる. 
\cref{sec:future}では今後の課題について述べる. 

本稿を通して以下の記号を用いる. 

\begin{symb}
	\begin{itemize}
		\item 複素数$ \tau $であって$ \Im(\tau) > 0 $なるもの.
		\item 絶対値が$ 1 $未満の複素数$ q := e^{2\pi\iu\tau} $.
	\end{itemize}
\end{symb}

% --------------------------------------------------------------------------

\section{Ramanujanの\ruby{擬}{モック}テータ関数と不定値テータ関数} \label{sec:mock}

% --------------------------------------------------------------------------

Zwegersによる不定値テータ関数の研究の動機にはRamanujanの\ruby{擬}{モック}テータ関数がある. 
そこで本節ではRamanujanの\ruby{擬}{モック}テータ関数の研究をかいつまんで紹介し, 不定値テータ関数とどのように関係するかを解説する. 

% --------------------------------------------------------------------------

\subsection{\ruby{擬}{モック}テータ関数からの研究の動機} \label{subsec:mock}

% --------------------------------------------------------------------------

Ramanujanがその最期にHardyへと宛てた手紙には, 彼が\textbf{\ruby{擬}{モック}テータ関数}と称する非常に奇妙な無限級数が記されていた. 
それがどのくらい奇妙なものかは実例を挙げれば一目瞭然である. 

\begin{dfn}[Ramanujanの位数$ 5 $の\ruby{擬}{モック}テータ関数]
	\[
	f_0(q) :=
%	\sum_{n=0}^{\infty} \frac{q^{n^2}}{(-q)_n}
%	=
	1 + \sum_{n=1}^{\infty} \frac{q^{n^2}}{(1+q)(1+q^2) \cdots (1+q^n)}.
	\]
\end{dfn}

%ただしここで$ (a)_n $はPochchammer記号で, 以下のように定義される. 
%
%\begin{dfn}[Pochhammer記号]
%	複素数$ a $と$ n \in \Z_{\ge 0} \cup \{ \infty \} $に対し
%	\[
%	(a)_n :=
%	\begin{cases}
%		1 & n=0, \\
%		(1 - a) (1 - aq) \cdots (1 - aq^{n-1}) & n \in \Z_{>0}, \\
%		(1 - a) (1 - aq) \cdots (1 - aq^{n-1}) \cdots & n = \infty.
%	\end{cases}
%	\]
%\end{dfn}

「今まで数学をやってきて, このような関数は見たことが無い」というのが私の率直な感想である. 
何しろ無限和の中に多項式の積が入り込んでいるのだ. 
このような関数を一体どのようにして研究できるというのだろうか?

だが投げ出す前に少し落ち着いて考えてみよう. 
「\ruby{擬}{モック}テータ関数」というRamanujanの命名は意味ありげである. 
このように名付けたからにはRamanujanはテータ関数に類似した性質を期待していたのだろう. 
しかしながら, 本来のテータ関数は正整数$ r $と正定値対称行列$ S \in \GL_r(\R) $に対して
\[
\theta_S (\tau) :=
\sum_{n \in \Z^r} q^{{}^t\!n S n/2}, \quad
q := e^{2\pi\iu\tau}
\]
と定義されるもので, Ramanujanの\ruby{擬}{モック}テータ関数とは似ても似つかない. 
果たして本当に類似した性質など期待できるだろうか?

だが「それでも何かありそうだ」と思わせるのがRamanujanである. 
もし仮に通常のテータ関数と\ruby{擬}{モック}テータ関数の間に類似した性質があるとすれば, それは何だろうか?

安直に思いつくのはテータ関数のモジュラー変換則である. 

\begin{prop}[テータ関数のモジュラー変換則, {\cite[命題2.5.1]{高瀬}}]
	\[
%	\theta_S (\tau + 1) = \theta_S (\tau), \quad
	\theta_S \left( -\frac{1}{\tau} \right) =
	\sqrt{\det S} \sqrt{-\iu \tau}^r \theta_S (\tau).
	\]
\end{prop}

この性質は非常に重要である. 
例えば最も簡単な$ r = 1, S = 1 $の場合にこの性質を用いることでRiemannゼータ関数の関数等式を得ることができる. 
またこの性質はテータ関数がモジュラー形式をなすことを意味しており, モジュラー形式の理論を応用することで様々な数論的帰結を得ることが可能である. 

そこで\textbf{\ruby{擬}{モック}テータ関数に対してもモジュラー変換則が満たされること}を期待してみたくなる. 
だがそれはいくら何でも楽観的過ぎるだろうか?

状況証拠はRamanujanの手紙に既に記されていた. 
Ramanujanは
\begin{itemize}
	\item \ruby{擬}{モック}テータ関数たちが非自明な関係式を満たすこと
	\item \ruby{擬}{モック}テータ関数のFourier係数の漸近的性質
\end{itemize}
を発見していたが, これらは共にモジュラー形式が持つ性質である. 
この期待を単に楽観的と片付けるのは勿体無さそうだ. 

状況証拠は\ruby{擬}{モック}テータ関数の研究が進むにつれて更に集まり始めた. 
例えばWatson~\cite[pp. 78]{Watson}は位数$ 3 $の\ruby{擬}{モック}テータ関数に対してモジュラー変換則を記述することに成功している. 
それは通常のテータ関数のような単純な形ではなく補正項を含んでおり, 非常に驚くべき式ではあるものの「確かに記述できるのは分かったけどこれで理解できたとは思えないなあ」と筆者には感じられる式である. 
Watsonの証明には拡張性に乏しく全ての\ruby{擬}{モック}テータ関数に適用することが難しいという問題もあった(彼の方法は\ruby{擬}{モック}テータ関数の「Appell--Lerch型和」と呼ばれる表示式に基づく巧妙なものである). 

更にその後, AndrewsやHickersonが\ruby{擬}{モック}テータ関数の「Hecke型表示」を発見した. 
それは例えば次のような表示式である. 

\begin{thm}[Andrews~{\cite[Equation (1.4)]{Andrews_5_7}}]
	\[
	f_0 (q) = \frac{1}{(q)_\infty}
	\sum_{n \ge 0} \sum_{-n \le j \le n} (-1)^j q^{(5n^2 - 2j^2 + n)/2} (1 - q^{4n+2}).
	\]
	ただし$ (q)_\infty := \prod_{n=1}^{\infty} (1 - q^n) $.
\end{thm}

この表示式で着目したいのが$ q $の肩に二次形式が乗っていることで, となると思い起こされるのはいつものテータ関数である. 
この見方は上の表示式を少し整えることで一層明白になる. 

\begin{cor} \label{cor:mock_indefinite}
	\[
	f_0 (q) = \frac{1}{(q)_\infty}
	\sum_{n, j \in \Z} \left( \sgn(j+n) - \sgn(j-n) \right) (-1)^j q^{(5n^2 - 2j^2 + n)/2}.
	\]	
	ただし実数$ x $に対し
	\begin{align}
		\sgn(x) &:= 
		\begin{cases}
			1 & x \ge 0, \\
			-1 & x < 0
		\end{cases}
	\end{align}
	と定める. 
\end{cor}

ここで符号などの不思議な項が現れているがそれは一旦置いておいて, $ q $の肩に乗っている二次形式への注目を続けよう. 
現れているのは$ 5n^2 - 2j^2 $という不定値な二次形式である. 
つまり右辺は\textbf{不定値テータ関数}と呼ぶべき対象が現れているのだ!
ここまで来ると\ruby{擬}{モック}テータ関数をテータ関数の類似物だとみなすことに確信が持ててくるだろう. 
そして
\begin{quote}
	\centering
	\ruby{擬}{モック}テータ関数はモジュラー変換則を持つか?
\end{quote}
という疑問は今や
\begin{quote}
	\centering
	不定値テータ関数はモジュラー変換則を持つか?
\end{quote}
という問いに変わったのである!

そこで次節以降ではこの二番目の問いについて考察していこう. 

% --------------------------------------------------------------------------

\subsection{補足: $ q $級数のEuler型表示式} \label{subsec:Eulerian_form}

% --------------------------------------------------------------------------

前項でRamanujanの\ruby{擬}{モック}テータ関数に対して「このような関数は見たことが無い」と書いたが, 実際にはこのようなタイプの無限和は$ q $級数の研究においてはRamanujan以前から現れており, 現在では「Euler型表示式 (Eulerian form)」と呼ばれている. 
\footnote{この用語はRamanujanによるもの(\cite[119ページ]{魅惑})であり, 恐らく\cref{thm:Jacobi}に由来する. 
	\cref{thm:Jacobi}はEulerによるものと述べられることが多いが, Andrews~\cite{Andrews_combi}によるとJacobiの仕事が初出だそうである
	(以上の歴史的背景は松坂俊輝氏に教わった). 
	そのような理由から私にはあまり良い用語だとは思えない. }
本項ではその例を紹介する. 

まずは$ q $級数の研究で頻出するPochchammer記号を準備しておこう. 

\begin{dfn}[Pochhammer記号]
	複素数$ a $と$ n \in \Z_{\ge 0} \cup \{ \infty \} $に対し
	\[
	(a; q)_n = (a)_n :=
	\begin{cases}
		1 & n=0, \\
		(1 - a) (1 - aq) \cdots (1 - aq^{n-1}) & n \in \Z_{>0}, \\
		(1 - a) (1 - aq) \cdots (1 - aq^{n-1}) \cdots & n = \infty
	\end{cases}
	\]
	と書く. 
\end{dfn}

それでは\ruby{擬}{モック}テータ関数に類する表示式を紹介しよう. 
まずは$ q $級数の中で最も古い歴史を持つ, 分割数の母関数から始める. 

\begin{thm}[Jacobi, {\cite[式 (8.2)]{整数の分割}}]
	\label{thm:Jacobi}
	\[
	\frac{1}{(q)_\infty} = \sum_{n=0}^{\infty} \frac{q^{n^2}}{(q)_n^2}.
	\]
\end{thm}

続けて紹介するのはRogers--Ramanujan恒等式と呼ばれる二つの等式である. 

\begin{thm}[Rogers--Ramanujan恒等式, {\cite[8.4節]{整数の分割}}]
	\begin{align}
		\frac{1}{(q; q^5)_\infty (q^4; q^5)_\infty} 
		&=
		\sum_{n=0}^{\infty} \frac{q^{n^2}}{(q)_n},
		\\
		\frac{1}{(q^2; q^5)_\infty (q^3; q^5)_\infty} 
		&=
		\sum_{n=0}^{\infty} \frac{q^{n^2 + n}}{(q)_n}.
	\end{align}
\end{thm}

この恒等式はこのように何気なく紹介されるとつい読み飛ばしてしまいそうになるが, 実際には深く考察すればするほど奥深い非常に重要な定理であり, 近年では丸一冊掛けてこの恒等式について解説する書籍が出版されたほどである(\cite{魅惑}). 

Euler型表示式はRamanujan以前のRogersの仕事にも表れている. 

\begin{thm}[{Rogers~\cite[pp. 333, Equation (6)]{Rogers}, \cite[240ページ, 式 (A.200)]{魅惑}}]
	\begin{align}	
		\sum_{n \in \Z} \sgn(n) q^{(3n^2 + n)/2}
		=
		\sum_{n=0}^{\infty} \frac{(-1)^n q^{n^2 + n}}{(-q)_n}.
	\end{align}
	ただしここで\cref{cor:mock_indefinite}と同様に$ \sgn(0) := 1 $と定義していることに注意する。
\end{thm}

この定理の左辺に現れる無限級数をRogersは\ruby{偽}{フォルス}テータ関数と呼んでいる。
正定値二次形式が$ q $の指数に現れながら符号も現れているのが特徴である。

Ramanujanはこれらの結果に知悉しており、非常に多くの変種を発見している。
\ruby{擬}{モック}テータ関数の発見にはそのような知的土壌があったものと推察される。

またこれら$ q $級数の研究からは, Ramanujanの\ruby{擬}{モック}テータ関数を研究するには\textbf{まず$ q $級数として取り扱う}ことで取り回しの良い表示式を得ることが出発点になると考えられるだろう. 
AndrewsやHickersonによって与えられた\ruby{擬}{モック}テータ関数の不定値テータ関数表示というのは正にそのような取り回しの良い表示式である. 
そしてそのような表示式が既に得られている現在では、Ramanujanによって与えられたEuler型表示式としての定義は最早忘れてしまい不定値テータ関数を新たな興味の対象として研究していく方針が取れる。
前項の最後に述べたのはまさにこのような方針である。

しかしながら疑問も残る。
本当にruby{擬}{モック}テータ関数の$ q $級数的側面を忘れてしまっても良いのだろうか?
本項で紹介した種々の公式はどれも非常に非自明で意味ありげであり、進路こそ見えないものの研究の余地は大いに残されているように思われる。
Zwegersによって\ruby{擬}{モック}テータ関数のモジュラー形式としての側面が解き明かされた今、次に研究すべきは$ q $級数としての側面であると私は感じている。

\begin{rem}
	Rogers--Ramanujan恒等式やRogersによる\ruby{偽}{フォルス}テータ関数の恒等式の変種は非常に多数発見されており, \cite[付録A]{魅惑}には計236個の変種が記載されている. 
	それら恒等式はどれも美しく, 非常に重要なものであるように思われるが, 一方で背景に潜む構造が判然としない散発的な等式群という印象も与えうる. 
	実際, それらの恒等式を証明するための強力な方法として\textbf{Bailey対 (Baliey pair)}という$ q $級数の手法があるが, これは発見的な方法であり, 背景にどのような構造が潜んでいるのかは説明できていないように思われる. 
	
	そのような問題点は「Euler型表示式」という用語にも潜んでいる. 
	この用語は数学的な定義が無くRamanujanの用法を参考に各々の数学者が各々の基準で用いているため, 「Euler型表示式である/ない」の判別が非常に曖昧なのである. 
	安直に考えればRogers--Ramanujan型の恒等式に登場する無限和は「Euler型表示式」と呼んで良いように思われるが, それら恒等式は後の時代に発見されたものになるほど非常に複雑な形をしており, 果たしてそもそも本当に「Rogers--Ramanujan型の恒等式」と呼んで良いものなのか段々分からなくなってくるのである. 
	このような問題はひとえにRogers--Ramanujan型の恒等式や\ruby{偽}{フォルス}テータ関数の恒等式の背景に潜む構造が未解明であることに起因していると思われる. 
	「Euler型表示式」がどのように数学的に定式化されるかは非常に意義深い問題であると私には感じられる. 
	このようなことも\ruby{擬}{モック}テータ関数のは$ q $級数的側面の研究と言えるだろう。
\end{rem}

% --------------------------------------------------------------------------

\section{不定値テータ関数のモジュラー変換則の探求} \label{sec:indefinite_theta}

% --------------------------------------------------------------------------






% --------------------------------------------------------------------------

\subsection{モジュラー補完という考え方} \label{subsec:modular_completion}

% --------------------------------------------------------------------------








% --------------------------------------------------------------------------

\subsection{} \label{subsec:}

% --------------------------------------------------------------------------














% --------------------------------------------------------------------------

\subsection{余談: Heckeの不定値テータ関数} \label{subsec:Hecke_theta}

% --------------------------------------------------------------------------

















% --------------------------------------------------------------------------

\section{Zwegersの不定値テータ関数} \label{sec:Zwegers_theta}

% --------------------------------------------------------------------------

% --------------------------------------------------------------------------

\subsection{不定値テータ関数の冪根での極限値} \label{subsec:indef_theta_limit}

% --------------------------------------------------------------------------

\cref{sec:WRT}でWRT不変量を重み付きGauss和で表示した.
この重み付きGauss和を冪根での極限値に持つような無限級数について考察しよう. 

二次形式$ Q(m, n) $が正定値の場合に重み付きGauss和が\ruby{偽}{フォルス}テータ関数の漸近展開として表せることは\cite{MM}で調べたが, $ Q(m, n) $が不定値の場合はそのような関数の存在は知られていない. 
%本項ではどのような関数を取り得るかについて考察する. 
そのような関数の候補としてまず初めに考えられるのがHeckeの不定値テータ関数である. 
正確な定義は述べないが, これは今の場合
\begin{align}
	\vartheta_Q (\tau) 
	&:=
	\sum_{l = {}^t\!(m, n) \in \Z^2 / \SL_2(\Z)_Q, \, Q(l) > 0} \sgn_0(m) q^{Q(l)}, 
	\\
	\vartheta_Q^* (\tau)
	&:=
	\sum_{l = {}^t\!(m, n) \in \Z^2 / \SL_2(\Z)_Q} \sgn_0(m) q^{\abs{Q(l)}}
\end{align}
のように定義されるものである. 
そこでこれら関数の$ \tau \to 1/k $での極限値や漸近展開を調べることを考えたいが, 和の添字に現れる$ l $が$ \Z_{\ge 0}^2 $のような分かりやすい集合(ベクトル空間内の錐)の形をしていないために極限値や漸近展開を調べることができず, Euler-Maclaurinの和公式(\cref{lem:Euler-Maclaurin})を適用することができない.
どのように極限値や漸近展開を調べるかを考えるのも面白いことのように思われるが, ここでは諦めることにして, 代わりにZwegersの不定値テータ関数(\cref{sec:Zwegers_theta})を考えることにする. 
Zwegersの不定値テータ関数は和の添字集合が錐となっているため極限値や漸近展開を調べることが出来るのである. 

% --------------------------------------------------------------------------

\subsection{Zwegersの不定値テータ関数の定義と\ruby{擬}{モック}モジュラー性} \label{subsec:Zwegers_theta_def}

% --------------------------------------------------------------------------

%前節までに行った考察により, 二次形式$ Q(m, n) $が不定値の場合にはWRT不変量を冪根への極限値に持つ級数としてZwegersの不定値テータ関数が候補に挙げられることが分かった. 
%本節ではこのことを確かめる. 

まずZwegersの不定値テータ関数の定義を述べる. 
なお本稿では二変数二次形式のみ扱うこととする. 
本節を通して以下の記号を固定する. 

\begin{symb}
	不定値対称行列$ S \in \Sym_2(\Z) $.
\end{symb}

これに対し次の記法を定めておく. 

\begin{dfn}
	不定値二次形式$ Q(l) := {}^t\!l S l /2, \, l \in \Z^2 $.
\end{dfn}

このときZwegersの不定値テータ関数は以下のように定義される. 

\begin{dfn}[{\cite[Equation 8.23]{BFOR}, Zwegers~\cite[Section 2.2]{Zwegers_thesis}}]
	\label{dfn:Zwegers_theta}
	\leavevmode %強制的な改行
	\begin{itemize}
		\item 不定値対称行列$ S \in \Sym_2(\Z) $,
		\item 不定値二次形式$ Q(n) := {}^t\!n S n, \, n \in \Z^2 $,
		\item ベクトル$ \lambda, \lambda' \in \R^2 $であって$ Q(\lambda), Q(\lambda'), {}^t\!\lambda S \lambda' < 0 $なるもの,
		\item ベクトル$ \gamma, \delta \in \R^2 $,
		\item ベクトル$ z = \gamma \tau + \delta \in \bbC^2 $
	\end{itemize}
	に対し
	\begin{align}
		\vartheta_{S, \lambda, \lambda'} \left( z; \tau \right)
		&:=
		\sum_{l \in \Z^2}
		\left(\sgn_0 \left( {}^t\!\lambda S (\gamma + l) \right) - \sgn_0( {}^t\!\lambda' S (\gamma + l) ) \right)
		\bm{e} \left( {}^t\!z S l \right) q^{Q(l)/2}
		\\
		&=
		\bm{e} \left( -{}^t\!\gamma S \delta \right) q^{-Q(\gamma)/2}		
		\sum_{l \in \gamma + \Z^2}
		\left(\sgn_0( {}^t\!\lambda S l ) - \sgn_0( {}^t\!\lambda' S l ) \right)
		\bm{e} \left( {}^t\!\delta S l \right) q^{Q(l)/2}
	\end{align}
	とおき, これを\textbf{Zwegersの不定値テータ関数}と呼ぶ. 
\end{dfn}

\begin{thm}[{\cite[Theorem 8.26]{BFOR}, Zwegers~\cite[Proposition 2.4]{Zwegers_thesis}}]
	Zwegersの不定値テータ関数を定める無限級数は収束する. 
\end{thm}

このように定義されたZwegersの不定値テータ関数は\cref{prop:infin_series_asymptotic}で極限値を計算した無限級数のような表示をしていないが, 次節でそのような表示に書き直す. 
ここではその前にZwegersの不定値テータ関数が持つ最も重要な性質である\ruby{擬}{モック}モジュラー性について見ていくことにする. 
それは次のように定式化される. 

\begin{thm}[{\cite[Theorem 8.30]{BFOR}}]
	\cref{dfn:Zwegers_theta}の設定下で$ \lambda, \lambda' \in \Z^2 $であり, それらの各成分は互いに素だと仮定する. 
	このとき$ \vartheta_{S, \lambda, \lambda'} \left( z; \tau \right) $は重さ$ 1 $のベクトル値混合\ruby{擬}{モック}モジュラー形式 (vector-valued mixed mock modular form) の成分となる. 
	特に$ \vartheta_{S, \lambda, \lambda'} \left( z; \tau \right) $はある合同部分群$ \Gamma \subset \SL_2(\Z) $に関する重さ$ 1 $の混合\ruby{擬}{モック}モジュラー形式である. 
\end{thm}

ここで混合\ruby{擬}{モック}モジュラー形式は以下のように定義される概念である. 

\begin{dfn}[{\cite[Definition 13.1]{BFOR}}]
	\textbf{重さ$ k $の混合調和Maass形式}%, もしくは\textbf{深さ$ 2 $, 重さ$ k $の調和Maass形式}
	とは有限和$ f_1(\tau) g_1(\tau) + \cdots + f_n(\tau) g_n(\tau) $で表される関数であって, 各$ f_i(\tau) $は重さ$ k_i $の弱正則モジュラー形式であり各$ g_i(\tau) $は重さ$ l_i $の制御可能増大度の調和Maass形式 (harmonic Maass form of manageable growth, \cite[Definition 4.1]{BFOR}) であり各$ 1 \le i \le n $に対し$ k_i + l_i = k $を満たすもののことである. 
	重さ$ k $の混合調和Maass形式の正則部分を\textbf{混合\ruby{擬}{モック}モジュラー形式}と呼ぶ. 
\end{dfn}

ここで定義から次が成り立つ. 

\begin{lem}
	重さ$ k $の混合\ruby{擬}{モック}モジュラー形式の冪根への極限値は深さ$ 2 $の量子モジュラー形式を定める. 
\end{lem}

従って二次形式$ Q(m, n) $が不定値の場合には, WRT不変量をZwegersの不定値テータ関数の冪根への極限値として表すことが出来ればその量子モジュラー性が分かることになる(後に\cref{rem:Zwegers_lim}でこれは実現できそうにないことが分かる). 
このことを実行するために, 次項ではZwegersの不定値テータ関数の別表示を求める. 

% --------------------------------------------------------------------------

\subsection{Zwegersの不定値テータ関数の別表示} \label{subsec:Zwegers_theta_rep}

% --------------------------------------------------------------------------

それではZwegersの不定値テータ関数の別表示を与えよう. 
ポイントは次の補題である. 

\begin{lem} \label{lembasis_indef}
	不定値対称行列$ S \in \Sym_2(\Z) $に対し$ \Z^2 $の元$ \lambda, \lambda' $と$ \Z^2 $の基底$ \widetilde{\lambda}, \widetilde{\lambda'} $であって
	$ {}^t\!(\lambda, \lambda') S (\widetilde{\lambda}, \widetilde{\lambda'}) $が対角行列となり, 
	$ Q(\lambda), Q(\lambda'), {}^t\!\lambda S \lambda' < 0 $なるものが存在する. 
\end{lem}

\begin{proof}
	単因子論より行列$ P, \widetilde{P} \in \GL_2(\Z) $であってある整数$ M \mid N $について
	\[
	{}^t\!P S \widetilde{P}
	= \pmat{M & 0 \\ 0 & N}
	\]
	を満たすものが存在する. 
	ここで
	\begin{align}
		\pmat{1 & 0 \\ 1 & 1}
		{}^t\!P S \widetilde{P}
		\pmat{1 & 0 \\ -N/M & 1}
		&=
		\pmat{M & 0 \\ 0 & N}, 
		\\
		\pmat{0 & 1 \\ \pm 1 & 0}
		{}^t\!P S \widetilde{P}
		\pmat{0 & \pm 1 \\ 1 & 0}
		&=
		\pmat{M & 0 \\ 0 & N}
	\end{align}
	であり$ \GL_2(\Z) $は$ \pmat{1 & 0 \\ 1 & 1} $と$ \pmat{0 & 1 \\ \pm 1 & 0} $で生成されるので, 任意の$ \gamma \in \GL_2(\Z) $に対しある$ \delta \in \GL_2(\Z) $が存在し
	$ {}^t\!(P \gamma) S \widetilde{P} \delta $が対角行列になるように出来る. 
	よって, ある負の整数$ a, b, c $によって
	\[
	{}^t\!(P \gamma) S P \gamma
	= \pmat{a & b \\ b & c}
	\]
	と書けるような$ \gamma \in \Gamma \cap \Mat_2(\Z) $が存在することを示せば,
	$ (\lambda, \lambda') := P \gamma, \, (\widetilde{\lambda}, \widetilde{\lambda'}) := \widetilde{P} \delta $
	とおくことで主張が従う. 
	以下
	\[
	\pmat{a & b \\ b & c}
	:= {}^t\!P S P
	\]
	とおく. 
	
	\textbf{Case 1}. $ a < 0 $のとき. 
	もし$ c > 0 $なら
	\[
	\pmat{1 & 0 \\ n & 1} \pmat{a & b \\ b & c} \pmat{1 & n \\ 0 & 1}
	=
	\pmat{a & an + b \\ an + b & a n^2 + 2bn + c}
	\]
	と計算できるので適切な$ n \in \Z $を取ることで$ c < 0 $として良い. 
	また
	\[
	\pmat{-1 & 0 \\ 0 & 1} \pmat{a & b \\ b & c} \pmat{-1 & 0 \\ 0 & 1}
	=
	\pmat{a & -b \\ -b & c}
	\]
	と計算できるので$ b < 0 $として良い. 
	よってこの場合には主張が示された. 
	
	\textbf{Case 2}. $ a > 0 $のとき. 
	Case 1より, ある$ \gamma = \pmat{ m & * \\ n & * } \in \SL_2(\Z) $が存在して$ a m^2 + 2b mn + c n^2 < 0 $を満たすことを示せば良い. 
	ここで
	\[
	a m^2 + 2b mn + c n^2
	=
	a \left( m + \frac{b}{a} n \right)^2 +\frac{\det S}{a} n^2
	\]
	なので$ a m^2 + 2b mn + c n^2 < 0 $という条件は$ \abs{ a m + b n } < (-\det S) \abs{n} $と同値である. 
	よってある$ \gamma = \pmat{ m & * \\ n & * } \in \SL_2(\Z) $が存在して
	\[
	n>0, \quad
	\frac{-b + \det S}{a} < \frac{m}{n} < \frac{-b - \det S}{a}
	\]
	を満たすことを示せば良い. 
	ここで有理数の稠密性からこの条件を満たす互いに素な$ m \in \Z, n \in \Z_{>0} $が取れる. 
	この$ m, n $に対し$ \gamma = \pmat{ m & * \\ n & * } $を満たす$ \gamma \in \SL_2(\Z) $が存在するので主張が従う. 
\end{proof}

\begin{rem} \label{rem:Zwegers_lim}
	\cref{lembasis_indef}より, ある$ \lambda, \lambda' \in \Z $と$ \Z^2 $の基底$ \widetilde{\lambda}, \widetilde{\lambda'} $と正の整数$ M, N $が存在して
	$ {}^t\!(\lambda, \lambda') S (\widetilde{\lambda}, \widetilde{\lambda'}) = \pmat{M & \\ & -N} $を満たし
	$ Q(\lambda), Q(\lambda'), {}^t\!\lambda S \lambda' < 0 $なるものが存在する. 
	このとき$ \gamma \in \left[ 0, 1\right) \lambda \oplus \left[ 0, 1\right) \lambda' $に対しZwegersの不定値テータ関数は
	\begin{align} 
		\vartheta_{S, \lambda, \lambda'} \left( \gamma \tau; \tau \right)
		&=
		\sum_{m, n \in \Z} \left( \sgn_0(m) + \sgn_0(n) \right) q^{Q(\gamma + m\widetilde{\lambda} + n\widetilde{\lambda'})/2}
		\\
		&=
		\left(
		2 \sum_{l \in \widetilde{\lambda} + \widetilde{\lambda'} + (\Z_{\ge 0} \widetilde{\lambda} \oplus \Z_{\ge 0} \widetilde{\lambda'})}
		-
		2 \sum_{l \in - \widetilde{\lambda} - \widetilde{\lambda'} + (\Z_{\ge 0} (-\widetilde{\lambda}) \oplus \Z_{\ge 0} (-\widetilde{\lambda'}))}
		\right.
		\\
		&+
		\left.
		\sum_{l \in \widetilde{\lambda} + \Z_{\ge 0} \widetilde{\lambda}}
		+
		\sum_{l \in \widetilde{\lambda'} + \Z_{\ge 0} \widetilde{\lambda'}}
		-
		\sum_{l \in -\widetilde{\lambda} + \Z_{\ge 0} (-\widetilde{\lambda})}
		-
		\sum_{l \in -\widetilde{\lambda'} + \Z_{\ge 0} (-\widetilde{\lambda'})}
		\right)
		q^{Q(\gamma + l)/2}
	\end{align}
	と書ける. 
	この表示式と\cref{prop:infin_series_asymptotic}で準備した漸近展開の公式および重み付きGauss和の消滅性(\cref{prop:Gauss_sum_vanish})から, Zwegersの不定値テータ関数を周期写像$ \veps $で重み付けて足し合わせたものは極限値が$ 0 $になることが分かる. 
%	\[
%	\lim_{t \to +0} \vartheta_{\gamma, \lambda, \lambda'} \left( \frac{1}{k} + t \iu \right)
%	= 0
%	\]
%	が従う. 
\end{rem}

\cref{rem:Zwegers_lim}よりWRT不変量をZwegersの不定値テータ関数の冪根への極限値として表すことは望めないことが分かった. 
そこで新しく\textbf{不定値\ruby{偽}{フォルス}テータ関数}を導入しよう. 

% --------------------------------------------------------------------------

\section{Zwegersの不定値テータ関数の定義の詳細} \label{sec:dfn}

% --------------------------------------------------------------------------



% --------------------------------------------------------------------------

\subsection{} \label{subsec:}

% --------------------------------------------------------------------------








% --------------------------------------------------------------------------

\subsection{} \label{subsec:}

% --------------------------------------------------------------------------












% --------------------------------------------------------------------------

\section{Vigneraの結果との関係} \label{sec:}

% --------------------------------------------------------------------------






% --------------------------------------------------------------------------

\section{} \label{sec:}

% --------------------------------------------------------------------------





% --------------------------------------------------------------------------

\section*{謝辞}

% --------------------------------------------------------------------------


% --------------------------------------------------------------------------
%		参考文献
% --------------------------------------------------------------------------

\bibliographystyle{alpha}
\bibliography{indefinite_theta}
% 日本語の書籍タイトルがゴシック体になる. 見苦しいようなら\emphコマンドを書き換える. 

% --------------------------------------------------------------------------
\end{document}
% --------------------------------------------------------------------------